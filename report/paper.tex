\documentclass[conference,10pt,letter]{IEEEtran}

\usepackage{url}
\usepackage{amssymb,amsthm}
\usepackage{graphicx,color}

\usepackage{float}

\usepackage{float}

\usepackage{cite}
\usepackage{amsmath}
\usepackage{amssymb}

\usepackage{color, colortbl}
\usepackage{times}
\usepackage{caption}
\usepackage{rotating}
\usepackage{subcaption}

\usepackage{balance}

\newtheorem{theorem}{Theorem}
\newtheorem{example}{Example}
\newtheorem{definition}{Definition}
\newtheorem{lemma}{Lemma}

\newcommand{\XXXnote}[1]{{\bf\color{red} XXX: #1}}
\newcommand{\YYYnote}[1]{{\bf\color{red} YYY: #1}}
\newcommand*{\etal}{{\it et al.}}

\newcommand{\eat}[1]{}
\newcommand{\bi}{\begin{itemize}}
\newcommand{\ei}{\end{itemize}}
\newcommand{\im}{\item}
\newcommand{\eg}{{\it e.g.}\xspace}
\newcommand{\ie}{{\it i.e.}\xspace}
\newcommand{\etc}{{\it etc.}\xspace}
%\newcommand{\em}[1]{\it}

\def\P{\mathop{\mathsf{P}}}
\def\E{\mathop{\mathsf{E}}}

\begin{document}
\sloppy
\title{Building radio map with geostatistics}
\maketitle
\begin{abstract}
Building a WiFi radio map is often useful to network engineers in order to understand 
the network covarage. In this short report we play a bit with geostatistics to 
create such radio map for an IEEE 802.11ac network deployed in an open-space.
\end{abstract}

\input intro.tex 
\input methodology.tex
\input results.tex
\input conclusions.tex

\balance
\bibliographystyle{abbrv}
\bibliography{mybib}

\end{document}
